\chapter{Important first chapter}
\label{chap:refs}

First chapter usually builds the theoretical background necessary for readers to understand the rest of the thesis. You should summarize and reference a lot of existing literature and research.

You should use the standard \emph{citations}\todo{Use \textbackslash{}emph command like this, to highlight the first occurrence of an important word or term. Reader will notice it, and hopefully remember the importance.}.

\begin{description}
\item[Obtaining bibTeX citation] Go to Google Scholar\footnote{\url{https://scholar.google.com}}\todo{This footnote is an acceptable way to `cite' webpages or URLs. Documents without proper titles, authors and publishers generally do not form citations. For this reason, avoid citations of wikipedia pages.}, find the relevant literature, click the tiny double-quote button below the link, and copy the bibTeX entry.
\item[Saving the citation] Insert the bibTeX entry to the file \texttt{refs.bib}. On the first line of the entry you should see the short reference name --- from Scholar, it usually looks like \texttt{author2015title} --- you will use that to refer to the citation.
\item[Using the citation] Use the \verb|\cite| command to typeset the citation number correctly in the text; a long citation description will be automaticaly added to the bibliography at the end of the thesis. Always use a non-breakable space before the citing parenthesis to avoid unacceptable line breaks:
\begin{Verbatim}
Trees utilize gravity to invade ye
noble sires~\cite{newton1666apple}.
\end{Verbatim}
\item[Why should I bother with citations at all?] For two main reasons:
\begin{itemize}
\item You do not have to explain everything in the thesis; instead you send the reader to refer to details in some other literature. Use citations to simplify the detailed explanations.
\item If you describe something that already exists without using a citation, the reviewer may think that you \emph{claim} to have invented it. Expectably, he will demand academic correctness, and, from your perspective, being accused of plagiarism is not a good starting point for a successful defense. Use citations to identify the people who invented the ideas that you build upon.
\end{itemize}
\item[How many citations should I use?]
Cite any non-trivial building block or assumption that you use, if it is published in the literature. You do not have to cite trivia, such as the basic definitions taught in the introductory courses.

The rule of thumb is that you should read, understand and briefly review at least around 4 scientific papers. A thesis that contains less than 3 sound citations will spark doubt in reviewers.
\end{description}

There are several main commands for inserting citations, used as follows:
\begin{itemize}
\item \citet{knuth1979tex} described a great system for typesetting theses.
\item We are typesetting this thesis with \LaTeX, which is based on \TeX{} and METAFONT~\cite{knuth1979tex}.
\item \TeX{} was expanded to \LaTeX{} by \citet{lamport1994latex}, hence the name.
\item Revered are the authors of these systems!~\cite{knuth1979tex,lamport1994latex}
\end{itemize}

\section{Some extra assorted hints before you start writing English}

Strictly adhere to the English word order rules. The sentences follow a fixed structure with subject followed by a verb and an object (in this order). Exceptions to this rule must be handled specially, and usually separated by commas.

Mind the rules for placing commas:
\begin{itemize}
\item Use the \emph{Oxford comma} before `and' and `or' at the end of a longer, comma-separated list of items. Certainly use it to disambiguate any possible mixtures of conjunctions: \textit{`The car is available in red, red and green, and green versions.'}
\item Do not use the comma before subordinate clauses that begin with `that' (like this one). English does not use subordinate clauses as often as Slavic languages because the lack of a suitable word inflection method makes them hard to understand. In scientific English, try to avoid them as much as possible. Ask doubtfully whether each `which' and `when' is necessary --- most of these helper conjunctions can be removed by converting the clause to non-subordinate.

As an usual example, \xxx{\textit{`The sentence, which I wrote, seemed ugly.'}} is perfectly bad; slightly improved by \xxx{\textit{`The sentence that I wrote seemed ugly.'}}, which can be easily reduced to \textit{`The sentence I wrote seemed ugly.'}. A final version with added storytelling value could say \textit{`I wrote a sentence but it seemed ugly.'}
\item Consider placing extra commas around any parts of the sentence that break the usual word order, especially if they are longer than a single word.
\end{itemize}

Do not write long sentences. One sentence should contain exactly one fact. Multiple facts should be grouped in a paragraph to communicate one coherent idea. Paragraphs are grouped in labeled sections for a sole purpose of making the navigation in the thesis easier. Do not use the headings as `names for paragraphs' --- the text should make perfect sense even if all headings are removed. If a section of your text contains one paragraph per heading, you might have wanted to write an explicit list instead.

Every noun needs a determiner (`a', `the', `my', `some', \dots); the exceptions to this rule, such as non-adjectivized names and indeterminate plural, are relatively scarce. Without a determiner, a noun can be easily mistaken for something completely different, such as an adjective or a verb.

Consult the books by \citet{glasman2010science} and \citet{sparling1989english} for more useful details.
