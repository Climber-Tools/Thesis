\chapter{Cled quickstart}\label{apx:cled}

\section{Setting up}
First, download the latest version from the \href{https://github.com/Climber-Tools/Cled/releases}{Releases page}.

To use Cled, you need a~wall and holds dataset.
A dataset example, along with the specification, can be found in \verb|Models/Example|.
This dataset can then be imported to Cled either by \verb|File > New|, or by pressing
\verb|Ctrl+N|.

To generate your own dataset, use Clis \cite{clis} to generate the hold and wall models and then use \verb|Scripts/ClisImporter.py| to prepare them for usage.
Note that you will likely need to install a~number of Python packages~---~to do this, you can follow the Clis setup on the \verb|Scripts/requirement.txt| file (see appendix \ref{apx:clissetup}).

\section{Key bindings}

\subsection{Movement}

\begin{tabular}[]{@{}ll@{}}
\toprule
\verb|WSAD| & move \verb|↑/↓/←/→| \\
\verb|Space| & fly upward \\
\verb|Shift| & fly downward \\
\bottomrule
\end{tabular}

\subsection{UI}

\begin{tabular}[]{@{}ll@{}}
\toprule
\verb|Esc| & pause/cancel \\
\verb|Enter| & confirm \\
\verb|Q| or \verb|Tab| & open holds menu \\
right button on selected route (ROUTE) & open route settings menu \\
\bottomrule
\end{tabular}

\subsection{Editing}

\begin{tabular}[]{@{}ll@{}}
\toprule
left button & pick up/place the hovered/held hold \\
right button & select the hovered route \\
\verb|E| & toggle between NORMAL and EDITING \\
\verb|R| or \verb|Delete| & delete hovered hold \\
\verb|Ctrl+R| or \verb|Ctrl+Delete| & delete hovered route/route of held hold \\
middle button (EDITING) + mouse & rotate held hold \\
\verb|T|/\verb|B| & toggle hovered hold as ending/starting \\
\verb|Ctrl| + click (ROUTE) & toggle hovered hold as being in the route \\
wheel up/down & cycle filtered holds \\
\bottomrule
\end{tabular}

\subsection{Import/Export}

\begin{tabular}[]{@{}ll@{}}
\toprule
\verb|Ctrl+N| & open new dataset \\
\verb|Ctrl+O| & open existing Cled project \\
\verb|Ctrl+S| & save project \\
\verb|Ctrl+Shift+S| & save project as \\
\verb|Ctrl+Q| & quit \\
\bottomrule
\end{tabular}

\subsection{Capturing images}

\begin{tabular}[]{@{}ll@{}}
\toprule
\verb|Ctrl+P| & capture image \\
\verb|Ctrl+Shift+P| & capture image as \\
\bottomrule
\end{tabular}

\subsection{Lighting}

\begin{tabular}[]{@{}ll@{}}
\toprule
\verb|F| & toggle user light \\
\verb|Ctrl+F| & add new light at the user position \\
\bottomrule
\end{tabular}

