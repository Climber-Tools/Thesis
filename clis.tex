\chapter{Clis quickstart}\label{apx:clis}

\section{Setting up}

For the setup, you'll need to first install \mintinline{text}{pyenv}:

\begin{minted}{text}
curl https://pyenv.run | bash
\end{minted}

and configure your shell's environment: \url{https://github.com/pyenv/pyenv\#basic-github-checkout}.
You'll then need to install Python \mintinline{text}{3.7.x} (due to Metashape and Blender working only with older Python versions):

\begin{minted}{text}
pyenv install -v 3.7.12
\end{minted}

To set up the virtual environment, do (in this directory)

\begin{minted}{text}
pyenv virtualenv 3.7.12 clis
pyenv local clis
pyenv activate clis
pyenv exec pip install -r requirements.txt
pyenv exec bpy_post_install
\end{minted}

\section{Contents}

Before running the scripts (using \mintinline{text}{pyenv\ exec}), it is advisable
to check the configuration file \mintinline{text}{config.py}, since a lot of the values will most likely
differ from their default values for your particular setup. Each of the
respective folders contain a \mintinline{text}{README.md} that further explains
the usage of each of the scripts.

\begin{itemize}
	\item \mintinline{text}{01-scanning/} -- tools for scanning the holds.
	\item \mintinline{text}{02-processing/} -- tools for processing the hold images into models.
	\item \mintinline{text}{03-data/} -- hold data format specification + tools for managing the models.
\end{itemize}

\section{Usage}

\subsection{Using tasks}

To simplify the usage of Clis, various pre-programmed scripts have been added.
To create a single model using them, you can do the following:

\begin{itemize}
	\item \mintinline{text}{./scan_single.sh} to scan a single hold,
	\item \mintinline{text}{./move_from_camera.sh} to move the hold photos from the camera,
	\item \mintinline{text}{./convert_from_raw.sh} to convert the photos to a usable format,
	\item \mintinline{text}{./generate_models.sh} to generate the model and finally
	\item \mintinline{text}{./add_models.sh} to add the model to the \mintinline{text}{holds.yaml} file.
\end{itemize}

\subsection{Using scripts}

Run

\begin{minted}{text}
pyenv exec python 01-scanning/01-scan.py automatic 15 \
	&& pyenv exec python 01-scanning/02-copy.py camera \
	&& pyenv exec python 01-scanning/03-convert.py
\end{minted}

to automatically take 15 pictures of the hold, copy it from the camera
and convert it from raw. If you want to take them manually because you
don't have the turntable, replace \texttt{automatic} with
\texttt{manual}. Then run

\begin{minted}{text}
pyenv exec python 02-processing/01-model.py
\end{minted}

To generate the model. Finally, run

\begin{minted}{text}
pyenv exec python 03-data/01-add_models.py
\end{minted}

To add the information about the hold to the \texttt{holds.yaml}
dictionary.
