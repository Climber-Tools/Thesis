\chapwithtoc{Conclusion}\label{sec:conclusion}
Clis and Cled have been successfully tested on the Smíchoff bouldering wall, creating a dataset of 119 holds (in 4 hours), the textured wall model (in 5 hours) and virtually setting a number of routes from the wall (see figure \ref{fig:capture}).

% hack to include this in TOC
\setcounter{secnumdepth}{0}

\section{Future developments}
Future developments include implementing a web interface and a mobile application, so the exported routes can be interacted with by regular users.
Such interface should support the ideas outlined in the introduction, such as route filtering, liking/disliking and send videos.

Additional features that were beyond the scope of this project but will be implemented in the future are automatic hold placement from a photo of the wall using SIFT (described in section \ref{ch:fext}), „snapping to hole“ for holds that can only be placed in specific ways and a large number of quality-of-life improvements and bugfixes.
Laser scanning is also a potential improvement for model generation.
Furthermore, the option of using VR to increase setting efficiency will be explored.

Another possible use for the data produced by Clis and Cled is for the purposes of simulating the movements of a climber on the wall.

The end goal is to provide Clis and Cled as a service for climbing gyms all over the world, so climbers have an easier time picking where to go climbing and can better find people to go climbing with, and routesetters have an easier time of building higher-quality routes.

\section{Routesetter feedback}
To test the editor implementation functionality, two of Smíchoff's routesetters were asked to provide feedback regarding the editor usage.
The feedback was positive, with the main conclusion being that the editor would be suitable for climbing gyms with lower frequency of wall changes, since the editor adds a non-trivial overhead that smaller climbing gyms could not sustain.

% hack to include this in TOC
\setcounter{secnumdepth}{3}
