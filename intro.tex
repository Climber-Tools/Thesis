
\chapwithtoc{Introduction}

With the rising popularity of climbing, in no small part due to the addition of the sport to the Tokyo 2020 Summer Olympics \cite{olympics}, climbing gyms are seeing a steady increase in new climbers.
An obvious attraction to both climbers and route setters is being able to model and display the current way the holds are set up (the current „setting“), offering a great number of advantages, such as

\begin{itemize}
	\item archiving older settings for tracking trends like difficulty and style
	\item being able to view the current setting online before visiting the gym and seeing if they are suitable (and possibly opting to go somewhere else if not)
	\item filtering boulders by difficulty to find suitable ones
	\item setting community-made boulders (if a suitable editor exists)
	\item adding a social aspect by (dis)liking and commenting on boulders, adding send videos and beta hints, etc.
\end{itemize}

Models of boulders are gradually becoming used in competitive bouldering and certain gyms, lead mainly by the OnlineObservation team \cite{onlineobservation}.
Their approach is simple -- take photos of the wall with the holds already on it and use them to generate a 3D model.
This works really well for a one-time model generation, but becomes infeasible for a bouldering gym that replaces boulders periodically, as the model would have to be regenerated each time, which takes a significant amount of time and specialized equipment.
It is also difficult to individually highlight certain boulders and edit them if a change is made after the setting.

This thesis attempts to solve this problem by focusing on what actually changes from setting to setting -- the position of the holds on the wall.
The repeated scanning of the holds and the wall adds redundancy, which could be removed by a program to edit the placements of the holds.
This adds time initially, since models of the wall and the holds have to be created.
However, it saves it in the long term and poses a number of advantages, making it a viable option for comercial bouldering gyms.


TODO: results

TODO: chapter overview

%Introduction should answer the following questions, ideally in this order:
%\begin{enumerate}
%\item What is the nature of the problem the thesis is addressing?
%\item What is the common approach for solving that problem now?
%\item How this thesis approaches the problem?
%\item What are the results? Did something improve?
%\item What can the reader expect in the individual chapters of the thesis?
%\end{enumerate}
%
%Expected length of the introduction is between 1--4 pages. Longer introductions may require sub-sectioning with appropriate headings --- use \texttt{\textbackslash{}section*} to avoid numbering (with section names like `Motivation' and `Related work'), but try to avoid lengthy discussion of anything specific. Any ``real science'' (definitions, theorems, methods, data) should go into other chapters.
%\todo{You may notice that this paragraph briefly shows different ``types'' of `quotes' in TeX, and the usage difference between a hyphen (-), en-dash (--) and em-dash (---).}
%
%It is very advisable to skim through a book about scientific English writing before starting the thesis. I can recommend `\citetitle{glasman2010science}' by \citet{glasman2010science}.
