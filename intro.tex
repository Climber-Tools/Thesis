\chapwithtoc{Introduction}

With the rising popularity of climbing and bouldering, in no small part due to the addition of the sport to the Tokyo 2020 Summer Olympics \cite{olympics}, climbing gyms are seeing a steady increase in new climbers.
An obvious attraction to both climbers and route setters is being able to virtually view the current way the holds are set up (the current „setting“).
This offers a great number of advantages, such as

\begin{itemize}
	\item archiving older settings for tracking trends like difficulty and style
	\item being able to view the current setting online before visiting a gym and seeing if it is suitable (and possibly opting to go somewhere else if not)
	\item filtering boulders by difficulty and popularity
	\item setting community-made boulders (if an editor exists)
	\item adding a social aspect by (dis)liking and commenting on boulders, adding videos and beta hints, connecting with other climbers, etc.
\end{itemize}

Models of boulders are gradually becoming used in competitive climbing and certain gyms, lead mainly by the OnlineObservation team \cite{onlineobservation}.
Their approach is simple -- take photos of the wall with the holds already on it and use them to generate a 3D model.
This works really well for a one-time model generation, but becomes infeasible for a climbing gym that replaces boulders periodically, as the model would have to be regenerated each time, which takes a significant amount of time and specialized equipment for higher quality models.
It is also difficult to individually highlight certain boulders and edit them if a change is made after the modelling.

This thesis attempts to solve this problem by focusing on what actually changes from setting to setting -- the position of the holds on the wall.
The repeated scanning of the holds and the wall adds redundancy, which could be removed by a program to edit the placements of the holds.
This adds time initially, since models of the wall and the holds have to be created.
However, it saves it after repeated settings and offers the aforementioned advantages, making it a viable option for commercial climbing gyms.

A system for semi-automatic scanning of climbing holds has been developed to efficiently create large hold datasets, along with a workflow for modelling climbing wall interiors (Clis \cite{clis}).
This data is then used in a virtual route editor that can be used to efficiently model the holds from real wall (Cled \cite{cled}).

The theory behind the techniques used in the hold and wall generation are covered in section \ref{sec:theory}, while the realization of the project is discussed in section \ref{sec:realization}.
Appendices \ref{apx:clis} and \ref{apx:cled} provide a quickstart for the scanning and editing respectively.
