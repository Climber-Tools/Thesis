\chapter{Generating markers}

The markers were calculated and generated using Matthew Petroff's Python scripts, available under the CC0 1.0 license.

To calculate the number of valid targets run \verb|find_codes.py|. To generate them in the PDF rormat, run \verb|create_target_sheets.py|. Note that to change the parameters of the PDF, you must directly edit the top of the source code.

% TODO: include the code?

%The program can be used as a C++ library, the simplest use is demonstrated in \cref{lst:ex}. A demonstration program that processes demonstration data is available in directory \verb|demo/|, you can run the program on a demonstration dataset as follows:
%\begin{Verbatim}
%cd demo/
%./bin/cool_process_data data/demo1
%\end{Verbatim}
%
%After the program starts, control the data avenger with standard \verb-WSAD- controls.
%
%\begin{listing}
%\begin{lstlisting}
%#include <CoolSoft.h>
%#include <iostream>
%
%int main() {
%	int i;
%	if(i = cool::ProcessAllData()) // returns 0 on error
%		std::cout << i << std::endl;
%	else
%		std::cerr << "error!" << std::endl;
%	return 0;
%}
%\end{lstlisting}
%\caption{Example program.}
%\label{lst:ex}
%\end{listing}
